%%%%%%%%%%%%%%%%%
% This is an sample CV template created using altacv.cls
% (v1.3, 10 May 2020) written by LianTze Lim (liantze@gmail.com). Now compiles with pdfLaTeX, XeLaTeX and LuaLaTeX.
%
%% It may be distributed and/or modified under the
%% conditions of the LaTeX Project Public License, either version 1.3
%% of this license or (at your option) any later version.
%% The latest version of this license is in
%%    http://www.latex-project.org/lppl.txt
%% and version 1.3 or later is part of all distributions of LaTeX
%% version 2003/12/01 or later.
%%%%%%%%%%%%%%%%

%% If you need to pass whatever options to xcolor
\PassOptionsToPackage{dvipsnames}{xcolor}

%% If you are using \orcid or academicons
%% icons, make sure you have the academicons
%% option here, and compile with XeLaTeX
%% or LuaLaTeX.
% \documentclass[10pt,a4paper,academicons]{altacv}

%% Use the "normalphoto" option if you want a normal photo instead of cropped to a circle
% \documentclass[10pt,a4paper,normalphoto]{altacv}

\documentclass[8pt,a4paper,ragged2e,withhyper]{altacv}

%% AltaCV uses the fontawesome5 and academicons fonts
%% and packages.
%% See http://texdoc.net/pkg/fontawesome5 and http://texdoc.net/pkg/academicons for full list of symbols. You MUST compile with XeLaTeX or LuaLaTeX if you want to use academicons.

% Change the page layout if you need to
%\geometry{left=1.25cm,right=1.25cm,top=1.5cm,bottom=1.5cm,columnsep=1.2cm}
\geometry{left=1.25cm,right=1.25cm,top=1.5cm,bottom=1.5cm}
% The paracol package lets you typeset columns of text in parallel
\usepackage{paracol}
\usepackage{hyperref}
% Change the font if you want to, depending on whether
% you're using pdflatex or xelatex/lualatex
\ifxetexorluatex
  % If using xelatex or lualatex:
  \setmainfont{Times New Roman}
  \setsansfont{Lato}
  \renewcommand{\familydefault}{\sfdefault}
\else
  % If using pdflatex:
  \usepackage{mathptmx}
  \usepackage[defaultsans]{lato}
  % \usepackage{sourcesanspro}
  \renewcommand{\familydefault}{\sfdefault}
\fi

% Change the colours if you want to

% \definecolor{SlateGrey}{HTML}{2E2E2E}
% \definecolor{LightGrey}{HTML}{666666}
% \definecolor{DarkPastelRed}{HTML}{450808}
% \definecolor{PastelRed}{HTML}{8F0D0D}
% \definecolor{GoldenEarth}{HTML}{E7D192}
% \colorlet{name}{black}
% \colorlet{tagline}{PastelRed}
% \colorlet{heading}{DarkPastelRed}
% \colorlet{headingrule}{GoldenEarth}
% \colorlet{subheading}{PastelRed}
% \colorlet{accent}{PastelRed}
% \colorlet{emphasis}{SlateGrey}
% \colorlet{body}{LightGrey}

% Change some fonts, if necessary
% \renewcommand{\namefont}{\Huge\rmfamily\bfseries}
% \renewcommand{\personalinfofont}{\footnotesize}
% \renewcommand{\cvsectionfont}{\LARGE\rmfamily\bfseries}
% \renewcommand{\cvsubsectionfont}{\large\bfseries}
\renewcommand{\namefont}{\LARGE\rmfamily\bfseries}
\renewcommand{\personalinfofont}{\footnotesize}
\renewcommand{\cvsectionfont}{\large\rmfamily\bfseries}
\renewcommand{\cvsubsectionfont}{\small\bfseries}

% Change the bullets for itemize and rating marker
% for \cvskill if you want to
\renewcommand{\itemmarker}{{\small\textbullet}}
\renewcommand{\ratingmarker}{\faCircle}

%% sample.bib contains your publications
\addbibresource{sample.bib}

\begin{document}
\name{WANG YANNAN, Dustin}
\tagline{}
%% You can add multiple photos on the left or right
%\photoR{2.6cm}{Dustin2-.png}
% \photoL{2.5cm}{Yacht_High,Suitcase_High}

\personalinfo{%
  % Not all of these are required!
  \email{dustin.yannan.wang@gmail.com}
  \phone{+852 53431852}
  \mailaddress{Room 805, Ho Sin Hang Engineering. Bldg, CUHK, Sha Tin, N.T., HK}
  \location{Hong Kong}
  \homepage{personal.ie.cuhk.edu.hk/~wy016/}
  %\twitter{@twitterhandle}
  %\linkedin{your_id}
  %\github{your_id}
  %% You MUST add the academicons option to \documentclass, then compile with LuaLaTeX or XeLaTeX, if you want to use \orcid or other academicons commands.
  % \orcid{0000-0000-0000-0000}
  %% You can add your own arbtrary detail with
  %% \printinfo{symbol}{detail}[optional hyperlink prefix]
  % \printinfo{\faPaw}{Hey ho!}[https://example.com/]
  %% Or you can declare your own field with
  %% \NewInfoFiled{fieldname}{symbol}[optional hyperlink prefix] and use it:
  % \NewInfoField{gitlab}{\faGitlab}[https://gitlab.com/]
  % \gitlab{your_id}
}

\makecvheader
%% Depending on your tastes, you may want to make fonts of itemize environments slightly smaller
% \AtBeginEnvironment{itemize}{\small}

%% Set the left/right column width ratio to 6:4.
%\columnratio{0.6}

% Start a 2-column paracol. Both the left and right columns will automatically
% break across pages if things get too long.



\cvsection{Education}
%\cvevent{Ph.D. in Information Engineering}{Sept 2016 -- July 2021}{The Chinese University of Hong Kong}{CGPA: 3.991/4.000}
\EducationEntry{Ph.D. in Information Engineering}{Sept 2016 -- June 2021}{The Chinese University of Hong Kong}{CGPA: 3.991/4.000}
\begin{itemize}
\item \textbf{Research Topic}: Optimization of some Non-convex Functionals arising in Information Theory
\item \textbf{Main Research}: Forward and reverse-hypercontractive inequality for binary erasure channel; weighted-sum rate outer bound on computing the module-two sum of doubly symmetric binary sources; log-convexity of Fisher Information
\item \textbf{Expertise}: Extensive experience on a wide range of mathematical tools and numerical simulation techniques to determine extremizers of non-convex optimization problems closely related to probability  
\end{itemize}
\divider
%\cvevent{B.Eng in Information Engineering}{Sept 2012 -- July 2016}{The Chinese University of Hong Kong}{First-class honour: CGPA=3.678/4.000; Major GPA=3.902/4.000}
\EducationEntry{B.Eng in Information Engineering}{Sept 2012 -- July 2016}{The Chinese University of Hong Kong}{CGPA: 3.678/4.000}
\begin{itemize}
    \item \textbf{Ho \& Ho Foundation Admission Scholarship} (2012-2016): Full tuition and living expenses coverage for studying in CUHK; Awarded annually to three mainland students who are academically outstanding% but have financial difficulty studying in CUHK 
    \item \textbf{Dean's List} (2013-2016): Awarded to students who attain a year GPA of 3.50 or above
    \item \textbf{Head's List (Merit)} (2014-2016): Awarded to students who attain a year GPA of 3.3 or above and also rank top 10\% among all students in the same major/programme
    \item \textbf{Best (Research) Project Award} (2015): One of the best participants in Summer Undergraduate Research Internship Programme awarded by Faculty of Engineering
    \item \href{https://www.erg.cuhk.edu.hk/erg/Elite}{Engineering Leadership, Innovation, Technology and Entrepreneurship (ELITE) stream}
    \item \textbf{First-class honour}: Major GPA=3.902/4.000
\end{itemize}


\cvsection{Publications \& Conferences}
Remark: \textit{Author names are in alphabetical order in the publications.}
\begin{itemize}
\item Qinghua (Devon) Ding, Chin Wa (Ken) Lau, Chandra Nair and Yan Nan Wang, ``\textit{Concavity of output relative entropy for channels with binary inputs}'', manuscript to the International Symposium on Information Theory (ISIT 2021), Melbourne, Victoria, Australia, June 2021.
\item Michel Ledoux, Chandra Nair and Yan Nan Wang, ``\textit{Log-convexity of Fisher information along heat flow}'', manuscript to the International Symposium on Information Theory (ISIT 2021), Melbourne, Victoria, Australia, June 2021.
\item Max Costa, Chandra Nair, David Ng and Yan Nan Wang, "\textit{On the structure of certain non-convex functionals and the Gaussian Z-interference channel}", presented at the International Symposium on Information Theory (ISIT 2020), Los Angeles, June 2020.
\item Chandra Nair, Yan Nan Wang, "\textit{On optimal weighted-sum rates for the modulo sum problem}", presented at the International Symposium on Information Theory (ISIT 2020), Los Angeles, June 2020.
\item Chandra Nair, Yan Nan Wang, "\textit{Reverse hypercontractivity region for the binary erasure channel}", presented at the International Symposium on Information Theory (ISIT 2017), Aachen, June 2017.
\item Chandra Nair, Yan Nan Wang, "\textit{Evaluating hypercontractivity parameters using Information Measures}", presented at the International Symposium on Information Theory (ISIT 2016), Barcelona, June 2016.
\end{itemize}



%\cvsection{Core Skills}
% \begin{itemize}
%     \item \textbf{Mathematics}
%     \begin{itemize}
%     \item \textbf{Probability Theory}: Grade A in postgraduate course Theory of Probability (IERG 6300), including law of Large numbers, Central Limit Theorem and conditional expectation
%     \item \textbf{Stochastic Process}: Grade A in postgraduate course Advanced Stochastic Models (SEEM 5580), covering Poisson Process, Markov chains, Martingales and Brownian Motion; Served as teaching assistant in undergraduate course Introduction to Stochastic Processes (IERG 3300)
%     \end{itemize}
%     \item \textbf{Programming}
%     \begin{itemize}
%     \item \textbf{Matlab}: Verifying conjectured properties of optimizers for certain non-convex functions via Matlab optimization functions such as fmincon
%     \item \textbf{C}: Served as teaching assistant in undergraduate course (IERG 2080) on C in Fall 2017, Fall 2018 and Fall 2020 
%     \item \textbf{Python 3}: Solving matrix inequalities via CVXOPT package of Python3 to verify feasibility
%      \end{itemize}

%      \iffalse
%      \item \textbf{Algorithms}\\
%     \begin{itemize}
%     \item Grade A in postgraduate course Advanced Algorithms (CSCI 5160), covering
%     \item Grade A in Theory of Computational Complexity (CSCI 5170), 
%     \end{itemize}
%     \fi
% \end{itemize}

%% Switch to the right column. This will now automatically move to the second
%% page if the content is too long.


%\switchcolumn

\cvsection{Experience}
\ExperienceEntry{Teaching Assistant}{Sept 2016 -- Dec 2020}
Conducted tutorials on network protocols for undergraduates; Demonstrated complex mathematical concepts and engineering techniques to undergraduates
\smallskip

\ExperienceEntry{Part-time student helper}{Dec 2014 -- Apr 2015}
Worked as a team leader with three junior undergraduates; Created and maintained a course website by Moodle; Prepared in-class exercises, online exercises and teaching notes to assist professor in teaching
% \begin{itemize} %add time
%     \item \textbf{Teaching Assistant} (Sept 2016-Dec 2020): Conducted tutorials for undergraduate students; Answered students' questions on course content
%     \item \textbf{Part-time student helper} (Dec 2014-Apr 2015): 
% \end{itemize}


  




\cvsection{language}
English (Fluent) $\vert$ Cantonese (Fluent) $\vert$ Mandarin (Native) 
\iffalse



\fi







%% Yeah I didn't spend too much time making all the
%% spacing consistent... sorry. Use \smallskip, \medskip,
%% \bigskip, \vpsace etc to make ajustments.




\end{document}
